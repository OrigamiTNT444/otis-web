\chapter{Introduction}
The book is divided into algebra, combinatorics, and number theory.
We do not cover geometry, for which
\emph{Euclidean Geometry in Mathematical Olympiads}
\cite{ref:EGMO} already serves the role of ``comprehensive book''.

The twelve main chapters in this book are structured in to four sections.
\begin{itemize}
	\ii A \textbf{theoretical portion}, of varying length,
	in which relevant theorems or ideas are developed.
	Some of this material is new, but the majority of it is not.
	Most of it has been adapted, edited, and abridged
	from existing handouts that you can still find at
	\begin{center}
		\url{http://web.evanchen.cc/olympiad.html}.
	\end{center}
	In general, the theoretical material here
	tries to stick to the basics, rather than being comprehensive.

	\ii A couple \textbf{walkthroughs}.
	These are olympiad problems
	which are chosen to illustrate ideas,
	accompanied by an outline of the solution.

	When designing my lecture notes for OTIS,
	I wrote these walkthroughs with the idea of emulating a person.
	In a real classroom the student does not simply passively listen to solutions.
	The process is more interactive:
	the instructor walks a student through the example,
	but with a back-and-forth series of prepared questions.
	My hope with the walkthroughs is to simulate
	this as best I can with static text.

	\ii A series of \textbf{problems}.
	These problems cover a range of difficulties.
	But in general, the first half of the problems in each chapter
	are intended to be fairly accessible,
	perhaps at the level of IMO 1/4.
	The difficulty increases quickly after that,
	with the closing problem usually being quite challenging.

	\ii Full \textbf{solutions} to both
	the walkthroughs and problems.
	(Great for inflating page count!)
	Readers are encouraged to read solutions even to problems
	that they solved; comments, remarks, or alternate solutions
	frequently appear.
\end{itemize}
In addition, at the end of each part,
a handful of problems chosen from USA selection tests are given,
mostly for fun.

In general, I assume the reader has some minimal
experience with reading and writing proofs.
However, I nonetheless dedicated the first chapter
to some mathematical and stylistic comments
which may be helpful to beginners in proofs.
Readers with significant proof experience should
feel no shame in skipping this first chapter.

\section*{Contest abbreviations}
Many problems have a source quoted,
but there are a large number of abbreviations as a result.
We tabulate some of the abbreviations here.

\begin{description}
	\ii[AIME] American Invitational Math Exam,
	the qualifying exam for the USA national olympiad.

	\ii[EGMO] European Girl's Math Olympiad
	(not to be confused with \cite{ref:EGMO})

	\ii[ELMO] The ELMO is a contest held at the USA olympiad
	training camp every year,
	written by returning students for newcomers.

	The meaning of the acronym changes each year.
	It originally meant ``Experimental Lincoln Math Olympiad''
	but future names have included ``$e^{\log}$ Math Olympiad'',
	``End Letter Missing'', ``Ex-Lincoln Math Olympiad'',
	``English Language Master's Open''.  ``Ego Loss May Occur'',
	``vEry badLy naMed cOntest'', ``Eyyy LMaO''.

	\ii[ELMO Shortlist] A list of problems
	from which each year's ELMO is chosen.

	\ii[HMMT] Harvard-MIT Math Tournament,
	the largest collegiate math competition in the United States.
	The contest is held twice a year, in November and February.

	\ii[IMO] International Math Olympiad, the supreme
	high-school mathematics olympiad.

	\ii[IMO Shortlist] A list of about $30$
	problems prepared annually,
	from which the six problems of the IMO are selected by vote.

	\ii[Putnam] The William Lowell Putnam Mathematical Competition,
	an annual competition for undergraduate students studying in USA and Canada.

	\ii[RMM] Romanian Masters in Mathematics,
	an annual olympiad held in Romania in late February
	for teams with a strong performance
	at the International Mathematical Olympiad.

	\ii[TSTST] The embarrassingly named
	``Team Selection Test Selection Test''.
	Held in June each year, the TSTST
	selects students for the USA Team Selection Test.

	\ii[TST] Abbreviation for Team Selection Test.
	Most countries use a TST as the final step
	in the selection of their team for the International Math Olympiad.

	\ii[USAJMO] USA Junior Math Olympiad,
	the junior version of the national math olympiad for the United States
	(for students in 10th grade and below).

	\ii[USAMO] USA Math Olympiad,
	the national math olympiad for the United States.
\end{description}
